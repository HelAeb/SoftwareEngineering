\documentclass[11pt,a4paper]{report}
\usepackage[utf8]{inputenc}
\usepackage{amsmath}
\usepackage{amsfonts}
\usepackage{amssymb}
\usepackage{amsmath, amsfonts, amssymb, amsthm, color, ulem, graphicx}
\usepackage{geometry}
\geometry{verbose,a4paper,tmargin=30mm,bmargin=30mm,lmargin=27mm,rmargin=27mm}
\usepackage{color}
\usepackage{gensymb}
\usepackage{fancyhdr}
\pagestyle{fancy}
\usepackage[T1]{fontenc}
\usepackage{Sweave}
\SweaveOpts{concordance=TRUE}
\usepackage{grffile}
\usepackage{natbib}
\usepackage{float}
\renewcommand\thesection{\arabic{section}}
\lhead{Aebersold, Nikolic, Schoch}
\rhead{Software Engineering}
\fancyheadoffset{0.2cm}
\renewcommand{\footrulewidth}{0.4pt}


\begin{document}

\title{\textbf{Software Engineering} \\ \vspace{40pt} Group Project}
\author{Helena Aebersold (10-605-921), \\ Divna Nikolic (12-614-715), \\ Michèle Schoch (10-607-448)}
\maketitle

%%%%%%%%%%%%%%%%%%%%%%%%%%%%%%%%%%%%%%%%%%%%%%%%%%%%%%%%%%%%%%%%%%%%%%%%%%%%%%%%%%%%%%%%%%%%%%%%%%%%%%%%%%%%%%%%%%%%%%%%%%%%%%%%%%%%%%%%%%%%%%
%~~~~~~~~~~~~~~~~~~~~~~~~~~~~~~~~~~~~~~~~~~~~~~~~~~~~~~~~~~~~~~~~~~~~~~
% Preliminary Code Junk
%~~~~~~~~~~~~~~~~~~~~~~~~~~~~~~~~~~~~~~~~~~~~~~~~~~~~~~~~~~~~~~~~~~~~~~
%%%%%%%%%%%%%%%%%%%%%%%%%%%%%%%%%%%%%%%%%%%%%%%%%%%%%%%%%%%%%%%%%%%%%%%%%%%%%%%%%%%%%%%%%%%%%%%%%%%%%%%%%%%%%%%%%%%%%%%%%%%%%%%%%%%%%%%%%%%%%%

<<fig = F, echo = F, results = hide>>=

# Source function file (from wd)
source("./red_button.R")

@


\pagenumbering{arabic}

%%%%%%%%%%%%%%%%%%%%%%%%%%%%%%%%%%%%%%%%%%%%%%%%%%%%%%%%%%%%%%%%%%%%%%%%%%%%%%%%%%%%%%%%%%%%%%%%%%%%%%%%%%%%%%%%%%%%%%%%%%%%%%%%%%%%%%%%%%%%%%
%~~~~~~~~~~~~~~~~~~~~~~~~~~~~~~~~~~~~~~~~~~~~~~~~~~~~~~~~~~~~~~~~~~~~~~
% Introduction		
%~~~~~~~~~~~~~~~~~~~~~~~~~~~~~~~~~~~~~~~~~~~~~~~~~~~~~~~~~~~~~~~~~~~~~~
%%%%%%%%%%%%%%%%%%%%%%%%%%%%%%%%%%%%%%%%%%%%%%%%%%%%%%%%%%%%%%%%%%%%%%%%%%%%%%%%%%%%%%%%%%%%%%%%%%%%%%%%%%%%%%%%%%%%%%%%%%%%%%%%%%%%%%%%%%%%%%
\newpage
\section*{Introduction}
Economic models in theory suggests specific changes depending on variations of other variables. For instance, if GDP rises, productivity rises and therefore interest rate is rising as more people demand for money. The purpose of this report is to find and analyze empirical evidence for the model implications for the country Switzerland. The data consists of eleven variables on quarterly basis, listed in the table below. The report, on one hand, analyzes cross-correlations between GDP and the different money aggregates (MB, M1, M2, M3. On the other hand, the effects of monetary policy shocks to GDP and inflation will be examined. 
\newline
\begin{table}[H]
\centering
\caption{Data variables}
\label{tab: var}
\begin{tabular}{l|l}
	\textbf{abbreviation} & \textbf{variable name}\\\hline\hline
	Date & Dates\\
	CPI & Consumer Price Index \\
	i10Y & long-run interest rates (10 years) \\
	GDP & gross domestic product\\
	GDP\_DEF & GDP deflator \\
	COM & commoditiy price \\
	i3M & short-run interest rates (3 month Libor) \\
	M1 & money aggregate 1 \\
	M2 & money aggregate 2 \\
	M3 & money aggregate 3 \\
	MB & money base \\
	RER & real exchange rate \\
\hline
\end{tabular}
\end{table}
 

%%%%%%%%%%%%%%%%%%%%%%%%%%%%%%%%%%%%%%%%%%%%%%%%%%%%%%%%%%%%%%%%%%%%%%%%%%%%%%%%%%%%%%%%%%%%%%%%%%%%%%%%%%%%%%%%%%%%%%%%%%%%%%%%%%%%%%%%%%%%%%
%~~~~~~~~~~~~~~~~~~~~~~~~~~~~~~~~~~~~~~~~~~~~~~~~~~~~~~~~~~~~~~~~~~~~~~
% Dynamic Correlation
%~~~~~~~~~~~~~~~~~~~~~~~~~~~~~~~~~~~~~~~~~~~~~~~~~~~~~~~~~~~~~~~~~~~~~~
%%%%%%%%%%%%%%%%%%%%%%%%%%%%%%%%%%%%%%%%%%%%%%%%%%%%%%%%%%%%%%%%%%%%%%%%%%%%%%%%%%%%%%%%%%%%%%%%%%%%%%%%%%%%%%%%%%%%%%%%%%%%%%%%%%%%%%%%%%%%%%
\newpage
\section{Dynamic Correlation}

In this section the cross-correlations between GDP and the four money aggregates MB, M1, M2 and M3 are analyzed. The analysis considers eight lags and leads, meaning a time span of two years into the past as well as into the future is included. According to our expectation from theoretic economic models, a rise in the monetary base should trigger GDP growth as central bank policies have an impact on output. Further, GDP and the other money aggregates should have a positive correlation based on the assumption that a higher GDP rises demand for money.

\begin{figure}[H]
\caption{Dynamic correlation of GDP and the four money aggregates}
\label{dyn_corr}
\centering
<<fig = T, echo = F, height = 6, width = 12>>=
plot <- corr.plot(correlation_data_long)
print(plot)
@
\end{figure} \ \\\textbf{Interpretation}
\newline
Monetary base (MB) is positively correlated with GDP in the lags, suggesting that a high monetary base is rising output in the future. At the same time the money aggregates M1, M2 and M3 have a lower correlation with GDP which is in line with our expectation that the MB has the biggest impact on the output.
\newline
The correlation of GDP and MB is lower than the correlation of GDP and the three money aggregates for the leads. This finding supports the economic theory as well, as a higher output will lead to more production and investment in the short run and thus to a higher demand for money which after some lags can be seen in a higher M1, M2 and M3. 


%%%%%%%%%%%%%%%%%%%%%%%%%%%%%%%%%%%%%%%%%%%%%%%%%%%%%%%%%%%%%%%%%%%%%%%%%%%%%%%%%%%%%%%%%%%%%%%%%%%%%%%%%%%%%%%%%%%%%%%%%%%%%%%%%%%%%%%%%%%%%%
%~~~~~~~~~~~~~~~~~~~~~~~~~~~~~~~~~~~~~~~~~~~~~~~~~~~~~~~~~~~~~~~~~~~~~~
% sVAR model
%~~~~~~~~~~~~~~~~~~~~~~~~~~~~~~~~~~~~~~~~~~~~~~~~~~~~~~~~~~~~~~~~~~~~~~
%%%%%%%%%%%%%%%%%%%%%%%%%%%%%%%%%%%%%%%%%%%%%%%%%%%%%%%%%%%%%%%%%%%%%%%%%%%%%%%%%%%%%%%%%%%%%%%%%%%%%%%%%%%%%%%%%%%%%%%%%%%%%%%%%%%%%%%%%%%%%%
\newpage
\section{sVAR Model}
To analyse the effect of monetary policy on GDP and inflation, a five-variable sVAR model was chosen, including the following variables: output $\gamma_t$, inflation $\pi_t$, money aggreagte $m_t$, interest rate $r_t$ and commodity prices $c_t$. According to the information criteria, a lag of two, meaning half a year, was chosen.
\newline
The economic assumptions for the order of the variables are as follows: 
\newline
Output will not react immediately to a monetary shock due to stickiness. Firms might only be able to adjust their production after some time has passed which leads to a lagged response of output. Furthermore, a change in GDP will affect the other economic variables as well, for instance, unemployment or prices. Therefore, output should be at one of the first places of the sVAR's ordering.
\newline 
Variations in inflation might follow the output adjustments and eventually have an upward pressure on the price level due to costly production alterations. In consequence, this variable has to be placed after output but before any monetary policy instrument.
\newline
Commodity prices, however, are considered to be independent of movements in output or inflation. The  production  process  which  normally  depends  on  commodity  is  thus  hugely  affected by alterations in the commodity price index. That is why commodity prices are set as the very first variable.
\newline
The monetary aggregate M1 is rather endogenous and shocks to this variable have a vast impact on interest rates. Hence, the money aggregate is placed before the interest rate. 
\newline
Lastly, interest rates are considered to be entirely endogenous and responding to all changes of the aforementioned variables. Central banks react to economic movements so interest rates is set at the end.
\newline
\noindent Thus the ordering in our model is like this:
	\begin{eqnarray*}
		\{c_t, y_t, \pi_t, m_t, r_t \}
	\end{eqnarray*}




\begin{figure}[H]
\caption{Impulse response functions (IRF)}
\label{irf}
\centering
<<fig = T, echo = F, height = 6, width = 15>>=
plot <- list()
for (i in 1:length(irf_name)){
plot[[i]] <- white.theme.irf.plot(subset(irf_data_long, variable == irf_name[i]))
}
do.call(grid.arrange, plot)
@
\end{figure}


\noindent \textbf{Interpretation}
\newline
The reaction of output and inflation is positive for some quarters in both cases after a monetary policy shock. An unexpected one standard deviation increase in money leads to an increase of output (upper left of Figure \ref{irf}), lasting for approximately three and a half years (14 quarters) before the effect converges to zero. More money leads to cheaper investments, encouraging firms to obtain credits which in turn boosts outcome. A similar argumentation can be made for the impulse response function of inflation to a shock in the money gap (upper right in Figure \ref{irf}).  Movements of prices come along with changes in money aggregates, so a higher amount of money in the market ceteris paribus will lead to higher inflation. However, the effect is much higher than in the case of output and lasts less long (barely two years). The graphs in the lower part of Figure \ref{irf} suggest that both GPD and inflation rise in response to an unexpected increase in the interest rate. The temporary influence of this shock persists longer in the case of inflation, where a convergence to the steady state doesn't seem to happen even after five years (20 quarters). According to economic theory,  a  restrictive  monetary  policy  should  not  lead  to  inflation  or  output  pressure. This could be explained by the assumption of monetary policy authorities having more information or the empirically discussed price puzzle. In the former, one can argue that central banks with more information expecting an output and inflationary pressure will increase interest rate to dampen the effect. Prices would then rise after the monetary contraction (although less than without contraction) along with output. Thus, an increase of output and prices is observed despite a contractionary monetary policy. The magnitude of these two graphs is less distinct and rather small compared to the first discussed case of a monetary shock. In either case, longer term effects are not taken into account as there are only two lags included in the sVAR model.

\end{document}

