\documentclass[11pt,a4paper]{report}
\usepackage[utf8]{inputenc}
\usepackage{amsmath}
\usepackage{amsfonts}
\usepackage{amssymb}
\usepackage{amsmath, amsfonts, amssymb, amsthm, color, ulem, graphicx}
\usepackage{geometry}
\geometry{verbose,a4paper,tmargin=30mm,bmargin=30mm,lmargin=27mm,rmargin=27mm}
\usepackage{color}
\usepackage{gensymb}
\usepackage{fancyhdr}
\pagestyle{fancy}
\usepackage[T1]{fontenc}
\usepackage{Sweave}
\SweaveOpts{concordance=TRUE}
\usepackage{grffile}
\usepackage{natbib}
\usepackage{float}
\renewcommand\thesection{\arabic{section}}
\lhead{Aebersold, Nikolic, Schoch}
\rhead{Software Engineering}
\fancyheadoffset{0.2cm}
\renewcommand{\footrulewidth}{0.4pt}


\begin{document}

\title{\textbf{Software Engineering} \\ \vspace{40pt} Group Project}
\author{Helena Aebersold (10-605-921), \\ Divna Nikolic (12-614-715), \\ Michèle Schoch (10-607-448)}
\maketitle

%%%%%%%%%%%%%%%%%%%%%%%%%%%%%%%%%%%%%%%%%%%%%%%%%%%%%%%%%%%%%%%%%%%%%%%%%%%%%%%%%%%%%%%%%%%%%%%%%%%%%%%%%%%%%%%%%%%%%%%%%%%%%%%%%%%%%%%%%%%%%%
%~~~~~~~~~~~~~~~~~~~~~~~~~~~~~~~~~~~~~~~~~~~~~~~~~~~~~~~~~~~~~~~~~~~~~~
% Preliminary Code Junk
%~~~~~~~~~~~~~~~~~~~~~~~~~~~~~~~~~~~~~~~~~~~~~~~~~~~~~~~~~~~~~~~~~~~~~~
%%%%%%%%%%%%%%%%%%%%%%%%%%%%%%%%%%%%%%%%%%%%%%%%%%%%%%%%%%%%%%%%%%%%%%%%%%%%%%%%%%%%%%%%%%%%%%%%%%%%%%%%%%%%%%%%%%%%%%%%%%%%%%%%%%%%%%%%%%%%%%

<<fig = F, echo = F, results = hide>>=

# Source function file (from wd)
source("./red_button.R")

@


\pagenumbering{arabic}

%%%%%%%%%%%%%%%%%%%%%%%%%%%%%%%%%%%%%%%%%%%%%%%%%%%%%%%%%%%%%%%%%%%%%%%%%%%%%%%%%%%%%%%%%%%%%%%%%%%%%%%%%%%%%%%%%%%%%%%%%%%%%%%%%%%%%%%%%%%%%%
%~~~~~~~~~~~~~~~~~~~~~~~~~~~~~~~~~~~~~~~~~~~~~~~~~~~~~~~~~~~~~~~~~~~~~~
% Introduction		
%~~~~~~~~~~~~~~~~~~~~~~~~~~~~~~~~~~~~~~~~~~~~~~~~~~~~~~~~~~~~~~~~~~~~~~
%%%%%%%%%%%%%%%%%%%%%%%%%%%%%%%%%%%%%%%%%%%%%%%%%%%%%%%%%%%%%%%%%%%%%%%%%%%%%%%%%%%%%%%%%%%%%%%%%%%%%%%%%%%%%%%%%%%%%%%%%%%%%%%%%%%%%%%%%%%%%%
\newpage
\section*{Introduction}
Economic models suggests specific changes depending another variable changes. For instance, if GDP rises, productivit rises and therefore interest rate is rising as more people are issuing a credit which is equal to demand for money is rising. The vector autoregression model implies that only the past has an influence on future values. Its purpose is therefore based on past data to forcast deviation from the steady state depending one variable changes. It is the standard model for assessing economic models. Futhermore, structural vector autoregression model is also used to 
The purpose  this report is find and analyze empirical evidence for the model implications for the country Switzerland. The data consists of eleven variables on quarterly basis of Switzerland. The variables are listed in tab: var. The report on the one hand analyzes cross correlations between GDP and the different money aggregats (MB1 M1, M2, M3). Secondly, the report aims analyze the effect of monetary policy shocks on GDP and inflation. 

%untersuchen Schweizer Daten. Theorie Geldmengi stiegt, Produktion stiegt, Zinse stiegt...Structural/ Vector Autoregression: Mit heute Zukunft regression. Struktural setze Theorie, commodity price beeinflusst GDP,  (Zentralbank regierte.. Zins Reaktionsvariable). Suche empirischer Beweis für theoretische Zusammenhänge Aufgabe 1: Wie korrelieren mehr Geldmenge, mehr GDP.. in Theorie...


\begin{table}
\centering
\caption{Data variables}
\label{tab: var}
\begin{tabular}{l|l}
	\textbf{Abbreviation} & \textbf{variable name}\\\hline\hline
	X & Dates\\
	CPI & Consumer Price Index \\
	i10Y & long-run interest rates (10 years) \\
	GDP & gross domestic product\\
	GDP_DEF & GDP deflator \\
	COM & commoditiy price \\
	i3M & short-run interest rates (3 month libor) \\
	M1 & money aggregate 1 \\
	M2 & money aggregate 2 \\
	M3 & money aggregate 3 \\
	MB & money base \\
	RER & real exchange rate \\
\hline
\end{tabular}
\end{table}
 

%%%%%%%%%%%%%%%%%%%%%%%%%%%%%%%%%%%%%%%%%%%%%%%%%%%%%%%%%%%%%%%%%%%%%%%%%%%%%%%%%%%%%%%%%%%%%%%%%%%%%%%%%%%%%%%%%%%%%%%%%%%%%%%%%%%%%%%%%%%%%%
%~~~~~~~~~~~~~~~~~~~~~~~~~~~~~~~~~~~~~~~~~~~~~~~~~~~~~~~~~~~~~~~~~~~~~~
% Dynamic Correlation
%~~~~~~~~~~~~~~~~~~~~~~~~~~~~~~~~~~~~~~~~~~~~~~~~~~~~~~~~~~~~~~~~~~~~~~
%%%%%%%%%%%%%%%%%%%%%%%%%%%%%%%%%%%%%%%%%%%%%%%%%%%%%%%%%%%%%%%%%%%%%%%%%%%%%%%%%%%%%%%%%%%%%%%%%%%%%%%%%%%%%%%%%%%%%%%%%%%%%%%%%%%%%%%%%%%%%%
\newpage
\section*{Dynamic correlation}

Economic models show that money supply rises if GDP is rising. A higher GDP leads to more investment demand which in turn leads to a higher quantity of money.  
The past shows that the money base (MB) is positively correlated to GDP. The correlation is lagged which means that the effects of a higher GDP are effective with lag of two. 


\begin{figure}[H]
\caption{Dynamic correlation of xxxxxxxx}
\label{dyn_corr}
\centering
<<fig = T, echo = F, height = 6, width = 12>>=
plot <- corr.plot(correlation_data_long)
print(plot)
@
\end{figure}

Interpretation 

Monetary base (MB) is positively correlated with GDP for different lags suggesting that a high monetary base is rising output. At the same time the money aggregates M1, M2 and M3 have a lower correlation with GDP which is in line with our expectation that the MB has the biggest impact on the output. The correlation forecast based on past data for different lag shows that has a lower correlation with GDP than the three money aggreates. The correlation of M1, M2 and M3 with GDP is even higher than the on of MB. An explanation could be: A higher ouptut now will lead to more production and investment in the future and thus to a higher demand for money. After some lags the effect appears also in M1, M2 and M3.


%%%%%%%%%%%%%%%%%%%%%%%%%%%%%%%%%%%%%%%%%%%%%%%%%%%%%%%%%%%%%%%%%%%%%%%%%%%%%%%%%%%%%%%%%%%%%%%%%%%%%%%%%%%%%%%%%%%%%%%%%%%%%%%%%%%%%%%%%%%%%%
%~~~~~~~~~~~~~~~~~~~~~~~~~~~~~~~~~~~~~~~~~~~~~~~~~~~~~~~~~~~~~~~~~~~~~~
% sVAR model
%~~~~~~~~~~~~~~~~~~~~~~~~~~~~~~~~~~~~~~~~~~~~~~~~~~~~~~~~~~~~~~~~~~~~~~
%%%%%%%%%%%%%%%%%%%%%%%%%%%%%%%%%%%%%%%%%%%%%%%%%%%%%%%%%%%%%%%%%%%%%%%%%%%%%%%%%%%%%%%%%%%%%%%%%%%%%%%%%%%%%%%%%%%%%%%%%%%%%%%%%%%%%%%%%%%%%%
\newpage
\section*{SVAR Model}

In five-variable SVAR model the optimal lag of two (meaning half a year) was chosen.  The model includes the following variables:  output \gamma\textsubscript{t}, inflation \pi\textsubscript{t}}, money aggreagte m\textsubscript{t}, interest rate r\textsubscript{t} and commodity prices c\textsubscript{t}. 
The economic assumptions for these cases are as follows; Output will not react immediately to a monetary shock due to stickiness. Firms might only be able to adjust their production after some time has passed which leads to a lagged response of output. Furthermore, a change in GDP will affect the other economic variables as well for instance unemployment or prices. Therefore, output should be at the first place of the SVAR's odering of the SVAR model. Variations in inflation might follow the output adjustments and eventually have an upward pressure on the price level due to costly production alterations. In consequence, this variable has to be placed after output but before any monetary policy instrument. Commodity prices, however, are considered to be independent of movements in output or inflation. The  production  process,  which  normally  depends  on  commodity  is  thus  hugely  affected by alterations in the commodity price index. That is why commodity prices are set as first variable in part. The monetary aggregate M1 is rather endogenous and shocks to this variable have vast impact on interest rates. Hence, the money aggreagte is placed before interest rate. Lastly, interest rates are considered to be entirely endogenous and responding to all changes of the aforementioned variables. Central Banks react to economic movements so interest rates is set at the end.

\begin{figure}[H]
\caption{impuls response functions (IRF)}
\label{irf}
\centering
<<fig = T, echo = F, height = 6, width = 15>>=
plot <- list()
for (i in 1:length(irf_name)){
plot[[i]] <- white.theme.irf.plot(subset(irf_data_long, variable == irf_name[i]))
}
do.call(grid.arrange, plot)
@
\end{figure}




\end{document}

